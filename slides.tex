\documentclass{article}
\usepackage{amssymb,amsmath}
\usepackage{ifxetex,ifluatex}
\ifxetex
  \usepackage{fontspec,xltxtra,xunicode}
  \defaultfontfeatures{Mapping=tex-text,Scale=MatchLowercase}
\else
  \ifluatex
    \usepackage{fontspec}
    \defaultfontfeatures{Mapping=tex-text,Scale=MatchLowercase}
  \else
    \usepackage[utf8]{inputenc}
  \fi
\fi
\ifxetex
  \usepackage[setpagesize=false, % page size defined by xetex
              unicode=false, % unicode breaks when used with xetex
              xetex]{hyperref}
\else
  \usepackage[unicode=true]{hyperref}
\fi
\hypersetup{breaklinks=true, pdfborder={0 0 0}}
\setlength{\parindent}{0pt}
\setlength{\parskip}{6pt plus 2pt minus 1pt}
\setlength{\emergencystretch}{3em}  % prevent overfull lines
\setcounter{secnumdepth}{0}

\title{Jacob's Ladder}
\author{Arnaud Bailly - Christophe Thibaut}
\date{01/12/2011}

\begin{document}
\maketitle

\section{Category Theory Crash course}

\begin{itemize}
\item
  What is a category?
\end{itemize}
\section{``Well-known'' (or not so well-known) Categories}

\begin{itemize}
\item
  Some examples of Categories: Set, monoid, poset\ldots{}
\item
  Hask: Category of Haskell types
\item
  CAM: Categorical abstract machine, a practical compilation scheme for
  ML languages based on standard categorical constructs
\end{itemize}
\section{Warmer: Deconstructing a function categorically}

Rules: - Use only arrows and objects' names (ie. do not describe the
internal structure of objects) - Assume we work in a well-behaved
category - Refine a function till you drop

\section{Constructions In Categories}

\section{Product: Tupling objects and functions}

\begin{itemize}
\item
  Down the ladder: Define records and structures, apply functions on
  them
\end{itemize}
\section{Sums: Choosing between alternatives}

\begin{itemize}
\item
  Down the ladder: Error handling, making partial functions total
\item
  Eg. Exceptions in Java/C\#
\end{itemize}
\section{Special objects: 0 and 1}

\begin{itemize}
\item
  \emph{0}: Initial object, one and only one arrow from 0 to any other
  object
\item
  \emph{1}: Terminal object, one and only one arrow from any other
  object to 1
\item
  Using 1 as a \emph{selector} for ``elements'' in objects (eg. members
  of a set)
\item
  Uniqueness up-to isomorphisms
\end{itemize}
\section{Exponential}

\begin{itemize}
\item
  representing \emph{function abstraction} in a category
\item
  let A, B be objects, we can form B\^{}A the \emph{exponential object}
  and a function eval : B\^{}A x A -\textgreater{} B s.t. for any f: C x
  A -\textgreater{} B, f factorizes uniquely through eval:

  $f = ev ∘ (λf × ι)$
\item
  This also says that one can form λf: C -\textgreater{} B\^{}A the
  function that that transforms a two-argument function (here C x A
  -\textgreater{} B) into a function from one argument to another
  function. We cannot however leave A in the equation is it provides the
  basis for the computation of the limit (??)
\end{itemize}
\section{Categorical Abstract Machine}

\begin{itemize}
\item
  Representing computation within a category : The Categorical Abstract
  Machine (Cousineau et al., 1987) allows compiling \emph{strict}
  functional programs into an intermediate form suitable for compilation
  to low-level languages (eg. C, assembly\ldots{})
\item
  Rests on the concept of a cartesian closed category: A category with
  all limits and exponentiation.
\end{itemize}
\section{Case Study: A proxy HTTP server}

\begin{itemize}
\item
  Define an HTTP Service for retrieving content of files from a uid
  stored in a repository
\end{itemize}
\section{Constructions Over Categories}

\begin{itemize}
\item
  Up the ladder: applying categorical concepts to categorical constructs
\item
  Yes, the category of categories exists! (with some
  restrictions\ldots{})
\end{itemize}
\section{Functors}

\begin{itemize}
\item
  Mappings from objects to objects s.t. composition of functions is
  preserved f: a -\textgreater{} b =\textgreater{} Ff : Fa
  -\textgreater{} Fb
\item
  Down the ladder: Defining parametric data structures (eg. type
  constructors) like Lists, Sets, Streams\ldots{}
\item
  Exercise: The functor of functions with errors
\end{itemize}
\section{Applicative Functors}

\begin{itemize}
\item
  \textless{}\$\textgreater{} : a -\textgreater{} b -\textgreater{} (f a
  -\textgreater{} f b) : this is standard ``functor application''
\item
  \textless{}*\textgreater{} : f (a -\textgreater{} b) -\textgreater{}
  (f a -\textgreater{} f b) : allows working on multiple arity
  functions, \emph{push argument application inside functor}
\end{itemize}
\section{Monads}

\begin{itemize}
\item
  Article from Erik Meijer about LINQ:
  http://cacm.acm.org/magazines/2011/10/131398-the-world-according-to-linq/fulltext
\item
  LINQ uses monads' \emph{flatMap} (aka. \emph{bind}) to construct
  ``queries'' over various datatypes, including relational structures
\item
  the various keywords of the SQLish language (SELECT, FROM, WHERE) are
  just instances of (a -\textgreater{} m b) which are chained in the
  monad
\end{itemize}
\section{Natural transformations}

\begin{itemize}
\item
  Mappings \emph{between functors}
\item
  Down the ladder: Functions on lists that change the ``shape'' of the
  list without changing the elements in it (reverse, inits,
  tails\ldots{})
\item
  Down the ladder:
\item
  I/O on HTTP messages, REST interface of a service
\item
  A transformation need to be an isomorphism (ie. invertible), it may
  ``forget'' some aspects of the functors it maps
\end{itemize}
\section{Conclusion}

Relating categorical thinking to agile design principles: - \emph{Simple
design}: only arrows and (abstract) objects, no patterns, no frameworks,
no fancy coding tricks - \emph{No BDUF}: Designing functions first: What
will you do with your data/objects?, refrain from defining the
implementation first, abstract from details to prevent data lock-in -
\emph{Removing duplication}: Generalizing to higher-levels allow DRY to
apply not only on functions but also on control structures and ``shape''
of the system - Focusing on the \emph{Domain}: First define core arrows,
those that are part of the domain you are working on ; then apply
functors and transformations to \emph{lift} from core domain to a more
complex behaviours (error handling, logging, interfacing with the
outside world)

Caveats: - Do not talk of this conference to your local mathematician,
its hair might go white earlier than expected! - True Cat.Th. is
insanely complex, it has connections to all of mathematics (logic,
topology, algebra, analysis) and is very abstract with lot of weird
constructions

\section{Reference}

\begin{itemize}
\item
  Scalaz: Library in scala, provide a wealth of categorical
  constructions for scala programs
\item
  Category in java: representing cat.th. in JAva
\item
  Haskell: pioneer in applied cat.th.
\item
  http://hseeberger.wordpress.com/2011/01/31/applicatives-are-generalized-functors/
\item
  http://aabs.wordpress.com/2008/05/29/functional-programming-lessons-from-high-school-arithmetic/
\end{itemize}
\section{Credits}

\begin{itemize}
\item
  \href{http://en.wikipedia.org/wiki/File:Himnastigi.jpg}{Bath's Jacob's
  Ladder}
\item
  \href{http://upload.wikimedia.org/wikipedia/commons/thumb/a/ad/Augsburg\_Barf\%C3\%BC\%C3\%9Ferkirche\_013.jpg/822px-Augsburg\_Barf\%C3\%BC\%C3\%9Ferkirche\_013.jpg}{Augsburg
  Barfüßerkirche}
\end{itemize}

\end{document}
